% le caratteristiche richieste dall'università sono elencate qui: https://stem.elearning.unipd.it/mod/book/view.php?id=234&chapterid=46#modalita
% 12pt: font richiesto dall'università
% twoside: i margini interni ed esterni sono scambiati per le pagine "a sinistra" e "a destra"
% openright: i capitoli cominciano in pagine dispari ("a destra")
% extreport: supporta 12pt

\documentclass[12pt,a4paper,twoside,openright]{extreport}

\usepackage{amsmath}                            % per avere più controllo sulle equazioni 
\usepackage{enumitem}                           % per avere più controllo sulle enumerazioni
\usepackage[
    a4paper,
    top=2cm,bottom=2cm,
    outer=2cm,inner=3cm,
    includeheadfoot
]{geometry}                                     % margini richiesti dall'università
\usepackage{graphicx}                           % per le immagini
\usepackage[newfloat]{minted}                             % per il codice con la colorazione della sintassi
\usepackage{subcaption}                         % per le sotto didascalie
\usepackage[italian]{babel}
\usepackage{csquotes}                           % per le citazioni
\usepackage[backend=biber,style=numeric,natbib=true,sorting=nyt]{biblatex}
\usepackage{fontspec}
\usepackage{setspace}
\usepackage{fancyhdr}
\usepackage{etoolbox}
\usepackage{rotating}
\usepackage[pdfa]{hyperref}
\usepackage{hyperxmp}
\usepackage{caption}


\sloppy % per evitare che il testo in \verb finisca oltre i margini
\selectlanguage{italian}
\bibliography{bibliografia}
\setmainfont{Times New Roman} % carattere richiesto dall'università
\onehalfspacing % interlinea richiesta dall'università
\setlength{\skip\footins}{2\bigskipamount} % spazio tra testo e piè di pagina
\setlength{\footnotesep}{12pt} % spazio tra ogni nota a piè di pagina

\newcommand{\bchapter}[1]{ % comando per NON mostrare l'header nelle prime pagine di ogni capitolo
  \chapter[#1]{#1\thispagestyle{empty}}
  \pagestyle{plain}
}

\renewcommand\captionfont{\footnotesize}  %imposta la grandezza delle caption come le note a piè di pagina
%\captionsetup[listing]{skip=-10pt}        %diminuisce lo spazio tra il codice e la sua caption
\AtBeginEnvironment{figure}{\vspace{1\baselineskip}} % Aggiunge spazio prima dell'ambiente figure

\AtBeginEnvironment{listing}{\vspace{1\baselineskip}} % Aggiunge spazio prima dell'ambiente listing

\newenvironment{code}{\captionsetup{type=listing}}{}  % Setto l'ambiente per l'inserimento dei codici
\SetupFloatingEnvironment{listing}{name=Codice}


% questi valori vengono usati nella composizione del frontespizio
\title{Progetto Programmazione di Sistemi Embedded: Mobile AI}
\author{Alessandro Girlanda, Andrea Mutti, Umberto Bianchin}

\hypersetup{%
  pdftitle={Progetto Programmazione di Sistemi Embedded: Mobile AI}
  pdfauthor = {Alessandro Girlanda, Andrea Mutti, Umberto Bianchin},
  pdfapart=1, %set to 1 for PDF/A-1
  pdfaconformance=B,
}

\begin{document}
    \pagenumbering{roman}
    \pagestyle{empty} % per le prime pagine, non mostrare il numero di pagina

    \begin{titlepage}
    % solo il frontespizio deve essere simmetrico rispetto ai margini interno ed esterno
    \newgeometry{hmargin=2.5cm,vmargin=2cm}
        \begin{figure}
            \centering
            \begin{subfigure}[b]{0.4\textwidth}
                \includegraphics[width=\textwidth]{Immagini/logo_unipd}
            \end{subfigure}
            \hfill
            \begin{subfigure}[b]{0.3\textwidth}
                \includegraphics[width=\textwidth]{Immagini/logo_dei.png}
            \end{subfigure}
        \end{figure}
    
        \vspace*{\stretch{0.5}}
    
        \begin{center}
            \makeatletter % serve per poter usare \@...

            % NOTA: il Times New Roman non supporta il maiuscoletto.
            %\textbf{DIPARTIMENTO DI INGEGNERIA DELL'INFORMAZIONE}\\
            %\vspace*{\stretch{0.1}}
            %\textbf{CORSO DI LAUREA IN INGEGNERIA INFORMATICA}
    
            %\vspace*{\stretch{0.5}}
            \LARGE
            \textbf{\@title}

            \begin{figure}[ht]
                \centering
                \includegraphics[width=0.5\textwidth]{Immagini/App/icona_app.jpg}
            \end{figure}
    
            \vspace*{\stretch{1.5}}
            \large
            \begin{flushright}
                \textbf{Gruppo:} \\ Alessandro Girlanda \\  Andrea Mutti \\  Umberto Bianchin \\
            \end{flushright}
    
            \vspace*{\stretch{2}}
            \textbf{ANNO ACCADEMICO 2023-2024}
        
            \makeatother % serve dopo \makeatletter
        \end{center}
    \restoregeometry
\end{titlepage}

    \cleardoublepage
    

    \pagestyle{plain} % comincia a mostrare il numero di pagina

    \tableofcontents
    \listoffigures
    %\listoflistings
    \cleardoublepage % per assicurarsi che la numerazione araba cominci col primo capitolo

    \pagenumbering{arabic}

    \fancypagestyle{plain}{                       %setto l'intestazione della pagina
    \fancyhf{} 
    \fancyhead[RO, LE]{\thepage}
    \fancyhead[LO, RE]{\nouppercase{\leftmark}}
    \renewcommand{\headrulewidth}{0.4pt}
    }
    \setlength{\headheight}{14.5pt}

    \bchapter{NNAPI}
\section{Accelerazione Hardware}
La caratteristica principale dell’uso di NNAPI è l’accelerazione hardware. In particolare, uno smartphone è sicuramente dotato di una unità di elaborazione
centrale (CPU da ora in avanti), che esegue tutte le principali operazioni, sfruttando i registri, la cache, la ram ed eventuali storage. Grazie alla CPU
si possono svolgere tutte le operazioni di I/O, gestione memoria, grafica, gestione batteria, errori e quant’altro.
La CPU da sola, però, nella maggior parte dei casi non riesce a reggere quantità elevate di calcolo, in quanto si trova già occupata a svolgere operazioni
base date dall’OS (Android nel nostro caso, ma anche iOS per esempio) e varie operazioni di I/O e memory. Proprio per questa sua limitazione, nella grande
maggioranza degli smartphone moderni si trovano installati nella Motherboard anche altri componenti, in particolare una Scheda Video (GPU) ed eventualmente
altri moduli quali Digital Signal Processor (DSP) e Neural Processing Unit (NPU). 
Una scheda video, come suggerisce il nome, fornisce la potenza di calcolo necessaria a renderizzare ogni aspetto legato alla GUI dell’utente, ma non solo,
infatti grazie alla sua elevata potenza, può essere ampiamente sfruttata per tutte le operazioni che richiedono una grande quantità di dati da elaborare e
lo stesso può essere detto per DSP. 
Discorso a parte per le NPU. Si tratta di un processore “complementare” adatto a eseguire operazioni di calcolo prettamente incentrate sulle reti neurali.
Su ambienti mobile la prima comparsa è stata nel 2017, quindi relativamente recente, grazie alla presenza di una NPU sul chip kirin 970, presente nel Huawei
Mate 10 pro.
La sola integrazione di questo componente ha portato grandi vantaggi in termini di prestazioni e di efficienza energetica (si parla di un 50\% di efficienza
energetica guadagnata rispetto alla gamma precedente) rendendolo di fatto importante per tutti i prodotti successivi. Grazie a questa novità da parte di
Huawei anche i vari competitors iniziarono a produrre, ricercare e sperimentare il più possibile per stare al passo con il mercato, rendendo di fatto le NPU,
e il mondo dell’AI e delle reti neurali in generale, una realtà quotidiana presente su piccolo schermo.

Tornando quindi all’accelerazione hardware, il punto cardine è la possibilità di sfruttare tutti gli altri componenti oltre alla CPU per eseguire complesse
operazioni di calcolo e facilitare l’elaborazione di dati. 
In tema reti neurali, si ottiene un guadagno sotto più punti di vista per quanto riguarda operazioni di inferenza. Sicuramente si ottiene un incremento
nella velocità di risposta, in quanto il carico di lavoro viene efficientemente distribuito tra le varie unità di calcolo disponibili, e una ridotta
latenza generale, in quanto non c’è necessità di contattare un server esterno per richiedere dati.
Grazie a ciò non è nemmeno necessaria una connessione esterna, quale wifi o LTE, in quanto è possibile avere i dati in locale, rendendola di fatto
disponibile sempre e comunque in qualsiasi situazione a patto che ci sia batteria residua.
La questione energetica è un argomento piuttosto delicato, perché unità di elaborazione come GPU e DSP sono molto dispendiose in termini energetici e
spesso si deve ricorrere a compromessi.
Altro punto a sfavore, soprattutto per smartphone meno performanti, è la questione memoria. L’APK specifico potrebbe avere un peso di svariati GB e in
esecuzione la RAM potrebbe facilmente risultare piena, rischiando di portare il sistema in stati di rallentamenti o crash.
Un punto di forza è sicuramente la privacy, in quanto i dati rimangono all’interno del dispositivo.

\section{API Android Neural Networks}
L’API Android Neural Networks è una API pensata per eseguire operazioni pesanti, in termini di calcolo computazionale, per il machine learning su
dispositivi Android. In particolare la compatibilità SW è elevata, in quanto presente in ogni dispositivo con Android 8.1 (livello API 27) o successivo,
coprendo quindi più del 90\% dei dispositivi android attualmente in uso.

La NNAPI può essere sfruttata programmando in linguaggio C/C++ e sfruttando librerie di sistema di livello superiore come "Runtime NNAPI”.
Runtime NNAPI si tratta di una libreria condivisa tra un’app e i driver back-end con la particolarità di essere aggiornabile, per ricevere aggiornamenti
al di fuori del normale ciclo di rilascio di Android. Gli sviluppatori, quindi, possono correggere più facilmente bug presenti nel runtime e migliorarne
le compatibilità con i driver.

La vera potenza di queste API però, risiede nella possibilità di essere sfruttate da framework superiori come TensorFlow Lite, PyTorch o Caffe2 che
creano e addestrano reti neurali. Grazie a NNAPI è dunque possibile sfruttare modelli già definiti e allenati, avendo quindi un’abbondanza di dati a
disposizione.

\begin{figure}
    \centering
    \includegraphics[width=0.8\textwidth]{Immagini/nnapi_architecture.png}
    \caption{Architettura di sistema per l'API Android Neural Networks. Fonte \cite{NNAPI}}
    \label{fig:nnapiArchitecture}
\end{figure}

In figura \ref{fig:nnapiArchitecture} si riesce a visualizzare facilmente il flusso di lavoro. L’applicazione utilizza le librerie e i framework di machine
learning (come PyTorch o Runtime NNAPI) che a loro volta sfruttano le API di basso livello di NNAPI.
NNAPI comunica con l’astrazione HW del dispositivo per rete neurale, contattando i driver che si interfacciano all’hardware specifico (GPU, DSP ecc.).
Questo meccanismo efficace permette di dividere il lavoro in vari livelli di astrazione, in modo da isolare i compiti e ridurre problematiche, aumentando
la rapidità di risoluzione dei vari bug presenti. Questo meccanismo di divisione dei compiti è alla base di due importanti design pattern, in particolare
“Responsibility”, che permette di dividere le responsabilità ad ogni layer, e “Layered Architecture”, che come suggerisce il nome è un pattern architetturale
che come concetto alla base ha la suddivisione del software in vari strati. La tipologia sopra descritta è alla base della progettazione Android e Java.

Per sfruttare i calcoli tramite NNAPI è necessario prima creare un grafico diretto che definisca i suddetti calcoli. Questo grafico infatti, combinato con
i vari dati di input, forma il modello per la valutazione del runtime NNAPI. Vengono definite 4 astrazioni principali con NNAPI:
\begin{itemize}
    \item \textbf{Modello}: si tratta di un grafico di calcolo delle operazioni matematiche e di tutti i valori costanti appresi durante il processo
    di addestramento e sono operazioni specifiche delle reti neurali. Includono convoluzione bidimensionale\footnote{https://en.wikipedia.org/wiki/Convolution}
    (2D), attivazione logistica (sigmoid\footnote{https://en.wikipedia.org/wiki/Sigmoid\_function}), attivazione lineare rettificata\footnote{https://en.wikipedia.org/wiki/Rectifier\_(neural\_networks)}
    (ReLU) e altro ancora. Una volta creato correttamente, può essere utilizzato nuovamente in tutti i vari thread e le compilation. In NNAPI, un modello
    è rappresentato come un'istanza ANeuralNetworksModel\footnote{https://developer.android.com/ndk/reference/group/neural-networks?hl=it\#aneuralnetworksmodel}.
    La creazione di un modello è un'operazione sincrona.
    \item \textbf{Compilation}: rappresenta una configurazione per compilare un modello NNAPI in codice di livello inferiore. La creazione di una
    compilazione è un'operazione sincrona. Una volta creato correttamente, può essere riutilizzato in più thread ed esecuzioni. In NNAPI, ogni
    compilazione è rappresentata come un'istanza ANeuralNetworksCompilation\footnote{https://developer.android.com/ndk/reference/group/neural-networks?hl=it\#aneuralnetworkscompilation}.
    \item \textbf{Memoria}: rappresenta la memoria condivisa, i file mappati di memoria e buffer di memoria simili. L'utilizzo di un buffer di memoria
    consente al runtime NNAPI di trasferire i dati ai driver in modo più efficiente. In genere, un'app crea un buffer di memoria condiviso contenente
    tutti i tensori necessari per definire un modello. Puoi anche utilizzare i buffer di memoria per archiviare gli input e gli output per un'istanza di
    esecuzione. In NNAPI, ogni buffer di memoria è rappresentato come un'istanza ANeuralNetworksMemory\footnote{https://developer.android.com/ndk/reference/group/neural-networks?hl=it\#aneuralnetworksmemory}.
    \item \textbf{Esecuzione}: interfaccia per l'applicazione di un modello NNAPI a un insieme di input e per la raccolta dei risultati. L'esecuzione
    può essere eseguita in modo sincrono o asincrono. Per l'esecuzione asincrona, più thread possono attendere la stessa esecuzione. Al termine di questa
    esecuzione, tutti i thread vengono rilasciati. In NNAPI, ogni esecuzione è rappresentata come un'istanza ANeuralNetworksExecution\footnote{https://developer.android.com/ndk/reference/group/neural-networks?hl=it\#aneuralnetworksexecution}.
\end{itemize}

\begin{figure}
    \centering
    \includegraphics[width=0.9\textwidth]{Immagini/nnapi\_flow.png}
    \caption{Flusso di programmazione per l'API Android Neural Networks. Fonte \cite{NNAPI}}
    \label{}
\end{figure}

\section{Modello}
L'unità di calcolo fondamentale in NNAPI è il modello, il quale è definito da uno o più operandi e operazioni.
Gli operandi sono oggetti di dati utilizzati nella definizione del grafico, tra i quali costanti, input e output e nodi intermedi e possono essere di due tipi: scalari e tensori.
Uno scalare è un singolo valore nei vari formati booleano, interi, virgola mobile a 16, 32 bit.
I tensori, invece, sono array n-dimensionali. In NNAPI i tensori possono anche essere quantizzati a 8 o 16 bit.

\begin{figure}[ht]
    \centering
    \includegraphics[width=0.7\textwidth]{Immagini/operandi.png}
    \caption{Esempio di operandi per un modello NNAPI}
    \label{fig:operandi}    
\end{figure}

\subsection{Fasi operative}
In NNAPI, un'operazione specifica i calcoli da eseguire, costituita dai seguenti elementi:
\begin{itemize}
    \item un tipo di operazione (ad es. addizione, moltiplicazione, convoluzione);
    \item un elenco di indici degli operandi utilizzati dall'operazione per l'input;
    \item un elenco di indici degli operandi utilizzati dall'operazione per l'output.
\end{itemize}
Prima di aggiungere l'operazione, devi aggiungere al modello gli operandi utilizzati o prodotti da un'operazione.

L’utilizzo delle Neural Networks API avviene a basso livello, grazie a linguaggi come C/C++. In particolare C++ rispecchia la programmazione ad oggetti (come lo sono Java e Kotlin) e permette quindi di
definire veri e propri modelli come oggi in maniera intuitiva.
Per creare un modello, per esempio, basta definire due righe di codice:
\begin{center}
\texttt{ANeuralNetworksModel\* model = NULL;} \\ \texttt{ANeuralNetworksModel\_create(\&model);}\\
\end{center}
Dopo la creazione è necessario aggiungere i vari operandi attraverso \texttt{ANeuralNetworks\_addOperand()} e sfruttando le strutture dati fornite, come \textit{ANeuralNetworksOperandType}.

\section{Compilation}
In questa fase viene determinato su quali processori verrà eseguito il modello e chiede ai driver corrispondenti di preparare l’esecuzione, anche eventualmente generando codice macchina specifico per i processori scelti.
Da sottolineare la possibilità di scegliere esplicitamente quali dispositivi utilizzare e il rapporto di runtime tra l’utilizzo della batteria e la velocità di esecuzione.

\subsection{Esecuzione Sincrona VS Asincrona}
Alcune operazioni nel runtime NNAPI sono eseguite in maniera sincrona. Ciò significa che ogni operazione di questo tipo può rallentarne l’esecuzione, in quanto è necessario attendere la fine di un’esecuzione per iniziare
la successiva e viene ridotta la potenza innata del multithreading.

Di contro, sfruttare esecuzioni asincrone non sempre è conveniente, in quanto il dispositivo deve poter generare e sincronizzare tutti i vari thread, con latenza variabile. A volte è possibile avere ritardi di più di
500 microsecondi tra quando viene inviata la notifica di un thread al momento in cui viene associato un core. 

Da Android 10 in poi, NNAPI supporta esecuzioni di burst tramite un oggetto apposito. Le esecuzioni burst altro non sono che sequenze di esecuzioni della stessa compilazione che si verificano in rapida successione,
come la campionatura audio o l’acquisizione video. Gli oggetti burst quindi può portare ad esecuzioni più rapide, dove gli acceleratori rimangono in uno stato di prestazioni elevate per tutta la durata burst.

\subsection{Gestione errori}
Di norma, se si verifica un errore durante il partizionamento o un errore con i driver scelti, NNAPI ricorre alla implementazione della CPU per una o più operazioni ove possibile. Sfruttando framework come TensorFlow Lite
nella maggior parte dei casi può essere vantaggioso disabilitare la riserva della CPU e gestire gli errori con implementazioni delle operazioni ottimizzate dal client.

Se il partizionamento o la compilazione non vanno a buon fine, l’intero modello sarà provato sulla CPU, stessa cosa vale per un’operazione che continua a non riuscire. Questo ovviamente riduce notevolmente la velocità
di esecuzione ma aumenta le possibilità di successo, riducendo eventuali errori hardware. Da considerare, però, che alcune operazioni non sono nemmeno supportate dalla CPU.

È possibile quindi disabilitare il fallback della CPU e rimuoverla dall’elenco dei processori nel caso in cui non la si voglia utilizzare per operazioni del modello.

\subsection{Memoria}
Da sottolineare la possibilità, da Android 11 e successive, di sfruttare domini di memoria che forniscono interfacce allocatrici per memorie opache. Ciò permette alle applicazioni di passare dati e ricordi nativi del
dispositivo tra le varie esecuzioni, in modo che NNAPI non copi o trasformi i dati inutilmente quando eseguono esecuzioni consecutive sullo stesso driver.

Queste funzionalità sono destinate principalmente ai tensori interni ai driver con pochi accessi lato client, come tensori di stato. Per tensori con frequenti accessi alla CPU lato client, è necessario utilizzare pool di memoria condivisi.

\section{Tempo di esecuzione}
Il test delle prestazioni di un'app può essere effettuato misurando il tempo di esecuzione o tramite profilazione. Per misurare il tempo di esecuzione, si può utilizzare l'API di esecuzione sincrona o API a livello inferiore
come ANeuralNetworksExecution\_setMeasureTiming e ANeuralNetworksExecution\_getDuration. Queste API consentono di misurare il tempo di esecuzione su un acceleratore o nel driver, escludendo l'overhead come quello del runtime NNAPI
e dell'IPC.

Le informazioni di tempistica possono essere utili per ottimizzare le prestazioni dell'applicazione in produzione, anche se la raccolta di tali informazioni potrebbe influenzare le prestazioni stesse, da notarne in fase di
controllo e revisione dei dati. È importante notare che solo il driver è in grado di calcolare il tempo trascorso su se stesso o sull'acceleratore.
Per profilare l'applicazione, si può utilizzare Android Systrace, che genera automaticamente eventi systrace per profilare diverse fasi del ciclo di vita del modello e i vari livelli dell'applicazione, come Application,
Runtime, IPC e Driver.

Per generare i dati di profilazione NNAPI, è possibile utilizzare il comando systrace di Android e l'utilità parse\_systrace. Questi strumenti consentono di analizzare gli eventi systrace generati dall'applicazione e di ottenere
una visualizzazione tabellare dei tempi spesi nelle diverse fasi del ciclo di vita del modello e nei diversi livelli dell'applicazione.
Inoltre, Android 11 introduce miglioramenti nella qualità del servizio (QoS) per NNAPI, consentendo alle applicazioni di indicare le priorità relative dei propri modelli e di impostare timeout per la compilazione e l'inferenza del modello.

Infine, sono disponibili argomenti avanzati sull'utilizzo degli operandi NNAPI, come tensori quantizzati, operatori facoltativi e tensori di ranking sconosciuto. Il benchmark NNAPI è disponibile su AOSP per valutare latenza e accuratezza
dei modelli.

\subsection{Log}
NNAPI offre informazioni diagnostiche utili nei log di sistema, che possono essere analizzati utilizzando l'utilità logcat. È possibile attivare il logging dettagliato per fasi o componenti specifici impostando la proprietà
\textit{debug.nn.vlog} tramite adb shell. Questo consente di registrare informazioni dettagliate su fasi come la creazione di modelli, la compilazione, l'esecuzione e altro ancora.

Una volta attivato, il logging dettagliato genererà voci di log a livello di informazioni con un tag impostato sul nome della fase o del componente. Per controllare il livello dei messaggi di log mostrati da logcat si utilizza
la variabile di ambiente \textit{ANDROID\_LOG\_TAGS}. Questo è solo un filtro applicato a logcat e si deve comunque impostare la proprietà debug.nn.vlog su all per generare informazioni di log dettagliate




    \bchapter{PyTorch Mobile}
\section{PyTorch \& PyTorch Mobile}
PyTorch\cite{Pytorch} è un framework di deep learning open source, sviluppato inizialmente da \textit{Meta AI} e ora parte della \textit{Linux Foundation}.
Progettato con Python, viene usato per creare reti neurali e per progetti di apprendimento automatico,
combinando la libreria di machine learning \textbf{Torch}\cite{Torch} con un’API di alto livello basata su Python. Torch è famosa, specialmente nel campo
del Deep Learning, per fornire tool semplici e flessibili insieme a performance elevate; uno dei sui punti salienti è il grande supporto per le GPU, che 
contribuisce ad un allenamento più efficiente dei modelli di deep learning.
PyTorch fornisce innanzitutto un pacchetto Python per funzionalità ad alto livello, come l'elaborazione dei \textbf{tensori}\footnote{Array multidimensionale utilizzato per memorizzare dati. Nel campo del Machine Learning vengono usati per rappresentare e manipolare input, pesi e output.}
ed inoltre un così detto \textbf{TorchScript}, che permette di creare modelli da PyTorch che possono poi venire salvati e caricati in un processo dove non c'è alcuna 
dipendenza di Python.

PyTorch Mobile\cite{PyTorchMed}, introdotto per la prima volta nel 2019 alla \textit{PyTorch Developer Conference}, si riferisce ad un set di librerie e funzionalità fornite da PyTorch che
permettono allo sviluppatore di eseguire un modello PyTorch direttamente su dispositivi mobili, come smartphone e tablet.

\section{Caratteristiche Principali}
Le caratteristiche principali di PyTorch Mobile, così come scritto sul sito ufficiale\cite{PyTorchOfficial}, sono:
\begin{itemize}
    \item Disponibile per iOS, Android e Linux;
    \item Fornisce API per comuni compiti di pre-processing e integrazione necessari ad incorporare Machine Learning nelle applicazioni mobile;
    \item Supporta la libreria XNNPACK per le CPU Arm e integra QNNPACK per i kernel a 8 bit quantizzati;
    \item Fornisce un efficiente interprete mobile per Android e iOS;
    \item Supporterà a breve backend hardware come GPU, DSP e NPU.
\end{itemize}

XNNPACK\cite{XNNPACK} è una libreria altamente ottimizzata per accelerare le operazioni di reti neurali convoluzionali (CNN) e altre operazioni di reti neurali su hardware mobile,
mentre QNNPACK\cite{QNNPACK} (Quantized Neural Network PACKage) è progettata per accelerare le reti neurali quantizzate\footnote{La quantizzazione è un processo che riduce la precisione dei numeri usati nei calcoli di una rete neurale, da 32-bit a 8-bit o meno, riducendo così l'uso della memoria e migliorando le prestazioni senza sacrificare significativamente l'accuratezza.}
su hardware mobile.


\section{Da un modello PyTorch a PyTorch Mobile}
Il tipico flusso dalla creazione del modello in PyTorch all'implementazione sul dispositivo mobile può essere visionato in figura \ref{fig:pyworkflow}; di seguito
verranno spiegati i vari step. Il primo step è ovviamente quello di scrivere un modello PyTorch o utilizzarne uno preesistente e convertirlo in un modello per dispositivi mobile; vedremo
solamente come compiere questa azione, visto che è ciò che viene richiesto dal progetto.

\begin{figure}[ht]
    \centering
    \includegraphics[width=0.5\textwidth]{Immagini/workflow_pytorch.png}
    \caption{Workflow dal training al rilascio di un modello su piattaforma mobile. Fonte \cite{PyTorchOfficial}}
    \label{fig:pyworkflow}
\end{figure}

\subsection{Quantizzazione}
La quantizzazione\cite{Quantizzazione} è una tecnica utilizzata principalmente per accelerare la fase di inferenza nei modelli di machine learning,
rendendo più veloce l'elaborazione delle previsioni o delle decisioni basate sui dati. Questo metodo funziona riducendo la precisione dei numeri utilizzati;
questo serve per avere una rappresentazione del modello più compatta e per poter utilizzare operazioni vettoriali più efficientemente su molti hardware.
Partendo da un modello FP32 (Floating Point 32 bit), PyTorch supporta la quantizzazione INT8 (Integer 8 bit), ottenendo così una riduzione della dimensione
del modello e della necessità di memoria di 4 volte. Inoltre il supporto hardware per la computazione INT8 è solitamente 2 o addirittura 4 volte più veloce
rispetto a quella FP32. Queste tecniche si utilizzano principalmente nella fase di inferenza del modello ("only the forward pass is supported for quantized operators"),
e non durante la fase di addestramento.

PyTorch fornisce tre differenti modalità per la quantizzazione:
\begin{itemize}
    \item Eager Mode Quantization (beta)
    \item FX Graph Mode Quantization (prototipo)
    \item PyTorch 2 Export Quantization
\end{itemize}
\subsubsection{Eager Mode Quantization}
Questa funzionalità, ancora in versione beta, richiede che l'utente gestisca manualmente le fasi di quantizzazione e dequantizzazione.
Supporta solamente i moduli\footnote{In PyTorch, un modulo (classe torch.nn.Module) rappresenta un componente di un modello che incapsula uno o più strati, parametri e una funzione di forward che definisce come l'input viene trasformato in output. Possono avere uno stato, ossia dei pesi aggiornati durante il training,}
e non le funzioni\footnote{Spesso presenti nel modulo torch.nn.functional, sono operazioni stateless che eseguono calcoli specifici come attivazioni (ReLU, sigmoid), operazioni di pooling, e altre trasformazioni matematiche. Non mantengono uno stato.}.

\subsubsection{FX Graph Mode Quantization}
Al contrario del primo, questo modo di effettuare la quantizzazione è automatizzato. Migliora la Eager Mode aggiungendo il supporto per le funzioni,
anche se potrebbe essere necessario effettuare un refactor del modello per renderlo compatibile con la modalità FX.

\subsubsection{PyTorch 2 Export Quantization}
Questa è la nuova modalità di quantizzazione completa del grafo, e può essere utilizzata da una percentuale più elevata di modelli rispetto alla
modalità grafica FX, anche se presenta ancora limitazioni riguardo alcuni costrutti Python e richiede l'intervento dell'utente per supportare il
dinamismo nel modello esportato. Le caratteristiche principali sono:
\begin{enumerate}
    \item API programmabili per configurare come un modello viene quantizzato.
    \item UX (User Experience) semplificata per gli utenti e per gli sviluppatori backend, poiché è necessario interagire con un singolo oggetto, chiamato \textit{Quantizer}.
    \item Rappresentazione (opzionale) del modello quantizzato di riferimento che può rappresentare calcoli quantizzati con operazioni intere più vicine agli attuali calcoli quantizzati che avvengono nell'hardware.
\end{enumerate}

Ci sono poi tre tipi di quantizzazione supportati, è inoltre possibile vedere quali operatori sono compatibili con i tipi di quantizzazione in figura \ref{fig:quantizationCoverage}:
\begin{itemize}
    \item Quantizzazione dinamica: pesi quantizzati con \textit{activations}\footnote{Le attivazioni, nel contesto del machine learning,
    si riferiscono ai valori di output prodotti dai neuroni di una rete neurale durante il processo di forward pass, ovvero quando l'input viene
    elaborato attraverso i vari strati della rete fino a generare un output.} lette/salvate in floating point e quantizzate per i calcoli;
    \item Quantizzazione statica: pesi e activations quantizzati, è necessaria una fase di calibrazione per determinare i migliori parametri di quantizzazione dopo l'addestramento;
    \item Quantizzazione statica \textit{aware training}: pesi e activations quantizzati, i parametri sono modellati durante l'allenamento.
\end{itemize}

\begin{figure}
    \centering
    \includegraphics[width=0.8\textwidth]{Immagini/operator coverage.png}
    \caption{Operatori compatibili con i due diversi tipi di quantizzazione}
    \label{fig:quantizationCoverage}
\end{figure}
    
\subsection{Scripting e Tracing del modello}
Questi due step\cite{TracingVSScripting} servono per convertire un \textit{nn.Module} in un grafo in formato TorchScript.
\begin{itemize}
    \item \textbf{Tracing} usa il comando \texttt{torch.jit.trace()}, in cui si passano come argomenti il modello e un input d'esempio. L'input verrà processato
    dal modello e le operazioni eseguite verranno appunto tracciate e registrate in un grafo.
    \item \textbf{Scripting} usa il comando \texttt{torch.jit.script()} che prende in input il solo il modello, in questo caso il modello verrà ispezionato
    staticamente e da questa analisi verrà generato il codice TorchScript.
\end{itemize}
Viene usato prevalentemente il metodo Scripting, poiché cattura sia le operazioni che tutta la logica del modello, inoltre se l'esportazione dovesse fallire sarà
quasi sicuramente per una ragione ben definita (di conseguenza la modifica da apportare sarà chiara). Il Tracing viene preferito quando non si ha accesso al codice
e quindi non è possibile apportare modifiche. È possibile anche utilizzarli insieme.

\subsection{Ottimizzazione}
Grazie alla funzionalità \texttt{torch.utils.mobile\_optimizer.optimize\_for\_mobile}\cite{Ottimizzazione} si semplifica il processo di ottimizzazione
di modelli per garantire che funzionino in modo efficiente su piattaforme con risorse limitate, come gli smartphone e i tablet.
Il comando \texttt{torch.utils.mobile\_optimizer.optimize\_for\_mobile(script\_module, optimization\_blocklist=None, preserved\_methods=None, backend='CPU')} esegue
diverse operazioni in base ai parametri passati:
\begin{enumerate}
    \item \textit{script\_module}: un'istanza del modello TorchScript;
    \item \textit{optimization\_blocklist}: ottimizzazioni da escludere tra quelle disponibili;
    \item \textit{preserved\_methods}: lista di metodi che devono essere mantenuti quando si invoca \texttt{freeze\_module};
    \item \textit{backend}: tipo di dispositivo usato per eseguire il modello risultante (CPU di default, oppure "Vulkan" o "Metal").
\end{enumerate}

In caso non si passi una lista di ottimizzazioni da non eseguire, vengono eseguite tutte le seguenti:
\begin{itemize}
    \item Conv2D + BatchNorm fusion: combina operazioni Conv2d\footnote{Applica una convoluzione 2D su un segnale di ingresso composto da diversi piani di ingresso.}
    e BatchNorm2d\footnote{Applica la Batch Normalization ad un input 4D.} in un'unica operazione Conv2d, aggiornando i relativi pesi e bias.
    \item Insert and Fold prepacked ops: sostituisce le operazioni Conv2D e le operazioni lineari con le loro controparti preconfezionate, ottimizzando l'accesso alla memoria e l'esecuzione del kernel.
    \item ReLU/Hardtanh fusion: integra operazioni di ReLU\footnote{Applica la funzione rectified linear unit elemento per elemento} o
    Hardtanh\footnote{Applica la funzione HardTanh function elemento per elemento} con le operazioni Conv2D o lineari precedenti, sfruttando l'ottimizzazione hardware XNNPACK.
    \item Dropout removal: elimina i nodi di dropout\footnote{Durante l'addestramento, azzera casualmente alcuni elementi del tensore di input
    con probabilità p} dal modello quando il training è impostato su false.
    \item Conv packed params hoisting:sposta i parametri impacchettati delle convoluzioni al modulo radice, riducendo la dimensione del modello senza
    influire sui risultati numerici.
    \item Add/ReLU fusion: trova e combina operazioni di addizione seguite da ReLU in un'unica operazione \texttt{add\_relu}.
\end{itemize}

%\subsection{Salvataggio}

    \bchapter{TensorFlow}
\section{Introduzione}
TensorFlow è una libreria open source per l'apprendimento automatico, il calcolo numerico e altre attività di analisi statistica e predittiva. Questo
tipo di tecnologia, sviluppata e rilasciata da Google nel novembre 2015, rende l’implementazione di modelli di machine learning più semplice e veloce per
gli sviluppatori, assistendo nel processo di acquisizione dei dati, nella formulazione di previsioni su larga scala e nel successivo affinamento dei risultati.
Lo scopo principale di TensorFlow è la creazione e l’addestramento di reti neurali, che possono essere utilizzate per moltissime applicazioni, quali:
\begin{itemize}
    \item Classificazione delle immagini;
    \item Elaborazione del linguaggio naturale;
    \item Analisi delle serie temporali;
    \item Riconoscimento vocale;
    \item Sviluppo di soluzioni di visione artificiale;
    \item Ottimizzazione di chatbot;
    \item Sistemi di assistenza clienti automatizzati;
    \item Altri ancora.
\end{itemize}

La versatilità e il grande range di applicazioni rendono TensorFlow uno strumento veramente potente e, per questo, è il \textbf{motore di AI più utilizzato},
come si può osservare in figura \ref{fig:diagramma}.

\begin{figure}
    \centering
    \includegraphics[width=0.5\textwidth]{Immagini/logo_dei.png}
    \caption{Diagramma dei punteggi di utilizzo di vari framework nel 2018}
    \label{fig:diagramma}
\end{figure}

\section{TensorFlow Lite}
TensorFlow si può dire essere il “genitore” di TensorFlow Lite, introdotto da Google nel 2017, che non è altro che una versione ottimizzata per l’uso su
dispositivi embedded e mobili. TensorFlow Lite è un framework di deep learning open-source, che converte un modello TensorFlow pre-addestrato in un formato
specifico, che può essere ottimizzato per la velocità e l’archiviazione. Le sue caratteristiche principali sono:
\begin{itemize}
    \item Supporto per più piattaforme, coprendo dispositivi Android e iOS, Linux embedded e microcontrollori;
    \item Supporto per diversi linguaggi di programmazione, come Java, Swift, Objective-C, C++ e Python;
    \item Alte prestazioni, facendo ricorso all’accelerazione hardware e all’ottimizzazione del modello;
    \item Ottimizzato per l’apprendimento automatico sul dispositivo, affrontando 5 vincoli chiave:
 \begin{itemize}
        \item Latenza: TensorFlow Lite riesce ad eliminare il ritardo tra l’invio dei dati al server e il ricevimento della risposta. Infatti, il
        framework non necessita di inviare dati ad un server esterno (visto che l’apprendimento e l’elaborazione dei dati avvengono direttamente sul
        dispositivo) e, quindi, il tempo di predizione viene ridotto notevolmente;
        \item Privacy: E’ garantita la riservatezza massima nell’elaborazione di dati sensibili o personali che, infatti, non vengono mai condivisi con
        server remoti. Questo è possibile perché i dati non lasciano mai il dispositivo dell’utente, assicurando la privacy massima;
        \item Connettività: TensorFlow lite non necessita di una connessione internet permettendo lo svolgimento di attività di apprendimento e inferenza
        anche in situazioni in cui non è possibile o pratico avere una connessione stabile;
        \item Dimensioni del modello: I dispositivi mobili o embedded dispongono, generalmente, di risorse minori rispetto ai computer. La riduzione dello
        spazio di archiviazione e della necessità di risorse computazionali per un modello TensorFlow diventa, dunque, essenziale per l’esecuzione di
        algoritmi di Machine Learning sul dispositivo;
        \item Consumo energetico: L’elaborazione dei dati e l’inferenza di un modello di Machine Learning possono incidere pesantemente sulla durata della
        batteria di un dispositivo mobile. TensorFlow Lite, infatti, si occupa di gestire efficientemente il consumo di energia. Inoltre, l’eliminazione
        della necessità di trasmettere dati riduce ulteriormente il consumo energetico.
    \end{itemize}
\end{itemize}

\section{Workflow di sviluppo}
TensorFlow Lite, insieme a TensorFlow, offre tutti gli strumenti per la generazione e l’utilizzo di un modello di Machine Learning. Il workflow di
sviluppo è il seguente:
\begin{itemize}
    \item Generazione di un modello TensorFlow Lite: A questo scopo, si può decidere se utilizzare un modello di TensorFlow Lite esistente, creare un
    modello TensorFlow Lite da zero o convertire un modello TensorFlow in un modello TensorFlow Lite. In questo report, analizzeremo l’ultima delle 3
    opzioni poiché coerente con il progetto;
    \item Inferenza: Il processo di esecuzione di un modello TensorFlow Lite sul dispositivo per effettuare previsioni basate sui dati di input prende
    nome di inferenza;
    \item Analisi e miglioramento delle prestazioni: tramite dei delegati, utilizzare l’accelerazione hardware in modo che il modello incontri requisiti
    di efficienza elevati.
\end{itemize}

\subsection{Generazione di un modello TensorFlow Lite}
TensorFlow Lite rappresenta i modelli in uno speciale formato portatile efficiente noto come \textbf{FlatBuffers} (identificato dall'estensione del file .tflite ).
Questa scelta offre numerosi vantaggi rispetto al formato del modello di buffer del protocollo di TensorFlow come: dimensioni ridotte
(codice meno ingombrante) e inferenza più rapida (accesso diretto ai dati senza un ulteriore passaggio di analisi/decompressione) che consente a TensorFlow
Lite di funzionare in modo efficiente su dispositivi mobili, che hanno disponibilità di calcolo e memoria limitate.

Un modello TensorFlow Lite, inoltre, può contenere i cosiddetti “metadati” che forniscono una descrizione del modello leggibile dall’uomo e dei dati
interpretabili dalla macchina per la creazione automatica di pipeline di pre e post-elaborazione durante l’inferenza sul dispositivo.
Come è stato anticipato, i principali metodi di generazione di un modello TensorFlow Lite sono 3: 
\begin{enumerate}
    \item Riciclare un modello pre-esistente TensorFlow Lite con o senza metadati;
    \item Creare un modello TensorFlow Lite da zero tramite l’ausilio della libreria TensorFlow Lite Model Maker che semplifica il processo di training di
    un modello TensorFlow Lite utilizzando un set di dati personalizzato;
    \item Conversione di modello TensorFlow in un modello TensorFlow Lite utilizzando il convertitore TensorFlow Lite.
\end{enumerate}

Noi ci concentreremo su quest’ultima opzione, in quanto è interessante studiare i meccanismi con cui viene trasformato un modello TensorFlow
(adatto per dispositivi di vario genere e potenza) in un modello TensorFlow Lite (usato per dispositivi mobili e, quindi, molto più limitati in termini
di memoria e risorse di calcolo).

La \textbf{conversione dei modelli} TensorFlow nel formato TensorFlow Lite prevede percorsi diversi a seconda del contenuto del modello di Machine Learning.
Come primo passaggio di tale processo, infatti, è necessario valutare il modello per determinare se può essere convertito direttamente.
Questa valutazione determina se il contenuto del modello è supportato dagli ambienti di runtime TensorFlow Lite standard in base alle operazioni
TensorFlow che utilizza. Se il modello utilizza operazioni non presenti nel set supportato, vi è la possibilità di eseguire il refactoring del modello
o utilizzare delle tecniche di conversione avanzate.

\begin{figure}
    \centering
    \includegraphics[width=0.5\textwidth]{Immagini/logo_dei.png}
    \caption{Workflow di conversione di TensorFlow Lite}
    \label{fig:workflow}
\end{figure}

La \textbf{valutazione della conversione} è un passaggio importante prima di provare a convertire il modello. Durante la valutazione, infatti,
si determina se il contenuto del modello è compatibile con il formato TensorFlow Lite in termini di dimensione dei dati utilizzati, requisiti di
elaborazione hardware e dimensioni e complessità del modello. 

La maggior parte dei modelli può essere convertita direttamente nel formato TensorFlow Lite ma alcuni modelli potrebbero necessitare di un refactoring o
di una conversione avanzata per renderli compatibili.

La conversione vera e propria avviene tramite il \textbf{convertitore TensorFlow Lite} che prende un modello TensorFlow e genera un modello TensorFlow Lite
in formato FlatBuffer. Il convertitore funziona con i seguenti formati del modello di input: SavedModel (modello TensorFlow salvato su disco come un
insieme di file), modello Keras (creato utilizzando l’api di alto livello Keras), formato Keras H5 (una variante del SavedModel supportato dall’api Keras)
e modelli creati tramite delle funzioni concrete (ossia tramite API TensorFlow di basso livello).

Il modello può essere convertito utilizzando l’\textbf{API Python} o anche lo strumento della riga di comando. E’ consigliato l’utilizzo dell’API Python perché
consente di integrare la conversione nella pipeline di sviluppo, applicare ottimizzazioni, aggiungere metadati e altre ulteriori attività che
semplificano il processo di conversione. La riga di comando, invece, supporta solo la conversione del modello di base.

L’API Python, in particolare usa la classe \textbf{TF.lite.TFLiteConverter} e il suo metodo \texttt{from\_saved\_model()}, \texttt{from\_keras\_model()} e \texttt{from\_concrete\_functions()}
per convertire modelli rispettivamente di tipo, SavedModel, Keras e da funzioni concrete.

Il convertitore accetta, inoltre, 3 opzioni (o flag) che personalizzano la conversione del modello:
\begin{itemize}
    \item I flag di compatibilità che specificano se la conversione deve consentire operatori personalizzati, nel caso in cui nel modello
    TensorFlow vi siano delle operazioni non supportate da TensorFlow Lite;
    \item I flag di ottimizzazione che specificano il tipo di ottimizzazione da applicare durante la conversione. Tendenzialmente la quantizzazione
    è la tecnica maggiormente usata;
    \item I flag di metadati che permettono l’aggiunta di metadati al modello convertito.
\end{itemize}
Nel caso in cui vi siano problemi di compatibilità con gli operatori, si può proporre la \textbf{conversione avanzata}, che prevede il refactoring del
modello e ulteriori opzioni alternative.

\subsection{Ottimizzazione del modello}
Per andare incontro alla limitatezza della memoria e della potenza di dispositivi mobili ed Edge, TensorFlow Lite fornisce delle \textbf{tecniche
di ottimizzazione} per far rientrare i modelli in questi vincoli.
I modi principali in cui l’ottimizzazione del modello aiuta lo sviluppo dell’applicazione sono:
\begin{itemize}
    \item Riduzione delle dimensioni: i modelli più piccoli dispongono di dimensioni di archiviazione ridotte sul dispositivo mobile dell’utente,
    dimensioni di download inferiori in termini di tempo e larghezza di banda e meno utilizzo della memoria RAM durante l’esecuzione, garantendo
    prestazioni e stabilità migliori.
    \item Riduzione della latenza: la diminuzione della quantità di tempo necessaria per eseguire una singola inferenza con un determinato modello
    (latenza) è sintomo di buone prestazioni. La latenza, inoltre, può avere impatto sul consumo energetico ed è importante, dunque, tenere questa
    caratteristica in considerazione;
    \item Compatibilità con l’acceleratore: l’ottimizzazione di un modello permette, in alcuni casi, di utilizzare acceleratori hardware estremamente
    efficienti e veloci.
\end{itemize}
L’ottimizzazione può, però, comportare modifiche nell’accuratezza del modello, che devono essere tenute in conto durante il processo di sviluppo di
un’applicazione. Queste variazioni di precisione dipendono molto dall’ottimizzazione del singolo modello e, per questo, non sono facili da prevedere.
In genere, però, i modelli ottimizzati su dimensioni o latenza perdono sempre una quantità variabile di precisione (tendenzialmente piccola). 
Ci sono diversi tipi di ottimizzazione, ma quelli più usati sono:
\begin{itemize}
    \item Quantizzazione: tecnica di ottimizzazione che funziona riducendo la precisione dei valori usati per rappresentare i parametri di un modello,
    che di default sono numeri in virgola mobile a 32 bit. In base ai requisiti dei dati, la richiesta di dimensione, la precisione e l’hardware supportato vi
    sono 4 tipi di quantizzazione: float16 post-training, gamma dinamica post-training, intera post-training e training consapevole della quantizzazione.
    Ogni tipologia è specifica per determinate casistiche ma, in generale, tutte le 4 quantizzazioni portano ad una riduzione della latenza e delle
    dimensioni a discapito della precisione del modello;
    \item Pruning: Metodologia di ottimizzazione che funziona rimuovendo i parametri all’interno di un modello che hanno solo un impatto minimo sulle
    sue previsioni. In questo modo il modello avrà le stesse dimensioni e la stessa latenza di runtime ma potrà essere compresso in maniera più efficace
    e, quindi, sarà più facile ridurre le dimensioni di download;
    \item Clustering: Strategia di ottimizzazione che prevede il raggruppamento dei pesi di ciascun livello di un modello in un numero predefinito di
    cluster e, per ogni gruppo, il calcolo del valore del centroide. In questo modo, si riduce il numero dei pesi e quindi si diminuisce la complessità.
    I modelli a cui è stata applicata questa tecnica possono essere compressi più efficacemente.
\end{itemize}

\subsection{Eseguire l'inferenza}
Una volta ottenuto il modello addestrato possiamo testarlo con operazioni di inferenza ossia il processo di generazione di stime del modello per nuovi
non usati per la fase di training. Nel caso di TensorFlow Lite l’inferenza può essere eseguita in due modi diversi in base al tipo di modello:
\begin{itemize}
    \item Nel caso di \textbf{modelli senza metadati} si può utilizzare l’API dell’interprete TensorFlow Lite;
    \item Nel caso di \textbf{modelli con metadati} è possibile far ricorso a API predefinite utilizzando le Task Library di TensorFlow Lite o
    costruire delle pipeline di inferenza personalizzate con le librerie di supporto di TensorFlow Lite.
\end{itemize}
Per eseguire un’inferenza in un modello TensorFlow Lite è necessario un \textbf{interprete}, il quale deve essere snello e veloce garantendo minima
latenza di carico, di inizializzazione ed esecuzione. L’inferenza in TensorFlow Lite segue i seguenti passaggi:
\begin{enumerate}
    \item Caricamento del modello che contiene il grafico di esecuzione;
    \item Costruzione di un interprete e trasformazione del formato dei dati di input grezzi in un formato supportato dal modello;
    \item Esecuzione dell’inferenza, utilizzando apposite API per l’allocazione dei tensori;
    \item Interpretare l’output in un modo significativo che sia utile nell’applicazione.
\end{enumerate}
Le API di inferenza di TensorFlow sono supportate da Android, iOS e Linux in più linguaggi di programmazione. L’esecuzione di un’inferenza (ma anche
la costruzione di modelli) può far uso di specifiche librerie che aiutano lo sviluppatore a creare esperienze Machine Learning migliori. 
Queste librerie si suddividono in: \textbf{task libraries} e \textbf{support libraries}.

\subsection{Task libraries di TensorFlow Lite}
Le task libraries di TFLite forniscono interfacce ottimizzate per modelli per attività frequenti di Machine Learning, come classificazione di immagini,
domande e risposte, ecc. Le interfacce sono progettate specificamente per ciascuna attività per ottenere le migliori prestazioni e usabilità. La libreria
attività funziona su più piattaforme ed è supportata su Java, C++ e Swift.
Quali sono le caratteristiche di una task library?
\begin{itemize}
    \item API ben definite utilizzabili anche da non esperti di machine learning: E’ possibile eseguire l’inferenza in sole 5 righe di codice. Le API
    fornite sono potenti e facili da usare, e permettono il facile sviluppo di modelli di machine learning su dispositivi mobili;
    \item Elaborazione dei dati complessa ma efficace: supporta la logica di elaborazione del linguaggio naturale per la conversione dei dati nel
    formato richiesto dal modello. Questa logica è usabile anche nell’addestramento e nell’inferenza;
    \item Aumento della performance: L’elaborazione dei dati viene eseguita in pochi millisecondi, assicurando inferenze rapide e poco costose;
    \item Estendibilità e personalizzazione: E’ possibile sfruttare tutte i vantaggi forniti dalle task libraries e creare facilmente API personalizzate
    di inferenza Android/iOS.
\end{itemize}

Grazie a queste efficienti librerie, sono supportate diverse attività tra cui: API di visione (come un classificatore di immagini o un rilevatore di
oggetti), di linguaggio naturale, di audio e personalizzate.

\subsection{Support libraries di TensorFlow Lite}
Gli sviluppatori di applicazioni per dispositivi mobili interagiscono con oggetti tipizzati come bitmap o con primitive come gli interi. TensorFlow Lite,
però, usa tensori nella forma di ByteBuffer che sono difficili da debuggare e manipolare. Le support libraries android di TensorFlow Lite nascono proprio
da questa necessità di supportare l’elaborazione di input e di output di modelli TFLite, rendendo l’interprete più facile da usare.
Le libreria di supporto TensorFlow Lite forniscono, per esempio, una serie di metodi di manipolazione delle immagini base (ritagli/ridimensionamenti) e
di elaborazione di dati audio di base.

\section{Miglioramento delle prestazioni: i delegati}
TensorFlow Lite mette a disposizione diverse strategie per l’ottimizzazione e la massimizzazione delle prestazioni di un modello di machine learning.
Uno di questi è il delegato. Un delegato consente di eseguire i modelli (in parte o interamente) su un altro esecutore più efficiente specificatamente
per il tipo di modello e la piattaforma su cui si esegue.                                                                                                                                       

I delegati abilitano l'accelerazione hardware dei modelli TensorFlow Lite sfruttando gli acceleratori sul dispositivo come la GPU e il processore di
segnale digitale (DSP).

Come impostazione di default TensorFlow Lite utilizza kernel CPU per l’ottimizzazione del set di istruzioni ARM Neon. Tuttavia, la CPU non è
necessariamente ottima per l'aritmetica complessa tipica dei modelli di machine learning. La maggior parte dei telefoni cellulari attuali contiene,
però, chip che sono in grado di gestire meglio queste operazioni pesanti. Utilizzarli per le operazioni di rete neurale offre enormi vantaggi in termini
di latenza ed efficienza energetica. Per esempio, le GPU riescono a velocizzare fino a 5 volte la latenza, mentre il processore DSP è in grado di ridurre
del 75\% il consumo di energia. A ciascuno di questi acceleratori sono associate API che consentono la computazione personalizzata, come OpenCL o OpenGL
ES per GPU e Hexagon SDK per DSP. Dunque, per il corretto funzionamento degli acceleratori è necessario scrivere parecchio codice. TensorFlow Lite risolve
il problema fornendo delle API che fungono come ponte tra il runtime TFLite e queste API di basso livello.

\begin{figure}
    \centering
    \includegraphics[width=0.5\textwidth]{Immagini/logo_dei.png}
    \caption{Distinzione tra kernel CPU e i delegati. Fonte}
    \label{fig:distinzione}
\end{figure}

\subsection{Scelta di un delegato}
TensorFlow Lite supporta più tipi di delegati, ognuno dei quali è ottimizzato per determinate piattaforme e particolari modelli. Nello specifico,
la scelta di un delegato si basa su due criteri principali: la piattaforma (Android o iOS) e il tipo di modello (a virgola mobile o quantizzato) da
accelerare. Per quanto riguarda la piattaforma:
\begin{itemize}
    \item Multipiattaforma (Android e iOS): GPU è l’unico tra i vari delegati che permette l’utilizzo sia su android che su iOS;
    \item Android: Ci sono due delegati che supportano l’utilizzo su android (e NON su iOS), ossia il delegato NNAPI (disponibile in android 8.1 e
    versioni successive) e il delegato Hexagon (disponibile in versioni android precedenti che non supportano NNAPI);
    \item iOS: L’unico delegato ottimizzato per iOS è Core ML (disponibile su dispositivi mobili apple con SoC A12 o superiore).
\end{itemize}
Per quanto riguarda il tipo di modello:
\begin{center}
    \begin{tabular}{ |c|c|c|c|c| }
        \hline
        \textbf{Tipo di modello} & \textbf{GPU} & \textbf{NNAPI} & \textbf{Hexagon} & \textbf{CoreML} \\
        \hline
        Virgola mobile (32 bit) & \checkmark  & \checkmark & X & \checkmark \\
        \hline
        Quantizzazione float16 post-training & \checkmark & X & X & \checkmark \\
        \hline
        Quantizzazione della gamma dinamica post-training & \checkmark & \checkmark & X & X \\
        \hline
        Quantizzazione intera post-training & \checkmark & \checkmark & \checkmark & X \\
        \hline
        Training consapevole della quantizzazione & \checkmark & \checkmark & \checkmark & X \\
        \hline
        
    \end{tabular}
\end{center}

Ogni acceleratore è progettato per una certa larghezza di bit dei dati. Per esempio, se viene fornito un modello in virgola mobile ad un delegato che
supporta solo operazioni quantizzate a 8 bit (come il delegato Hexagon), allora il delegato non accetterà il modello il quale verrà eseguito interamente
sulla CPU.

Scegliere la configurazione di accelerazione hardware ottimale per il dispositivo di ciascun utente può essere difficile. Inoltre, abilitare la
configurazione errata su un dispositivo può causare elevata latenza, errori di runtime o problemi di precisione causati da incompatibilità hardware
sfociando, quindi, in un servizio scadente e insoddisfacente per l’utente. In questo contesto, TensorFlow Lite mette a disposizione un \textbf{servizio di
accelerazione per Android}: un'API che aiuta nella scelta della configurazione di accelerazione hardware ottimale per un determinato dispositivo utente e
per uno specifico modello, riducendo al minimo il rischio di errori di runtime o problemi di precisione.

Il servizio di accelerazione valuta diverse configurazioni di accelerazione sui dispositivi target eseguendo benchmark di inferenza con il modello
TensorFlow Lite creato. I risultati dell’esecuzione dei benchmark possono essere salvati su cache e utilizzati durante l’inferenza. Inoltre, questi
benchmark sono fuori processo, riducendo al minimo il rischio di arresti anomali dell’applicazione Android.

\begin{figure}
    \centering
    \includegraphics[width=0.5\textwidth]{Immagini/logo_dei.png}
    \caption{Schema riassuntivo del funzionamento del servizio di accelerazione}
    \label{fig:accelerazione}
\end{figure}

Fornendo il  modello, i campioni di dati e i risultati attesi il servizio di accelerazione eseguirà un benchmark di inferenza TFLite per fornire
consigli hardware allo sviluppatore.

\subsection{Tools per la valutazione}
TensorFlow Lite fornisce strumenti di valutazione delle prestazioni e dell’accuratezza che consentono agli sviluppatori di analizzare e verificare
l’efficienza dell’utilizzo dei delegati nell’applicazione creata. I due principali tools utilizzati sono:
\begin{itemize}
    \item Tools per la latenza e l’utilizzo di memoria: che stimano le prestazioni del modello considerando la latenza media di inferenza,
    l’overhead di inizializzazione, l’ingombro di memoria e altri;
    \item Tools per l’accuratezza e la correttezza: tendenzialmente, i delegati eseguono calcoli con una precisione differente rispetto ai kernel CPU.
    Di conseguenza, quando uso un delegato per l’accelerazione hardware la precisione tende a diminuire. Questo non succede sempre: per esempio la GPU,
    che esegue operazioni in virgola mobile, potrebbe migliorare leggermente la precisione. I tools che misurano questa metrica possono essere basati o
    meno sulla task specifica da valutare.
\end{itemize}

    \bchapter{PyTorch Mobile vs TensorFlow Lite}
Sono stati singolarmente analizzati entrambi i framework, ma quale bisogna scegliere per l’implementazione di modelli di machine learning su dispositivi mobili?
Ci si limiterà a confrontare i due framework sulle caratteristiche più importanti per lo sviluppo di un modello:
\begin{itemize}
    \item Facilità di implementazione: un punto critico per ogni sviluppatore è quello di poter implementare facilmente un progetto e iniziare ad usarlo fin da subito senza troppe configurazioni e setup iniziali.
    In questo contesto, TensorFlow Lite offre un set vastissimo di APIs\cite{tensorvspyt} che permettono di semplificare il processo di implementazione in modo significativo. Tendenzialmente, TensorFlow Lite richiede solo 2 ore per
    implementare un semplice prototipo. È possibile implementare le stesse funzionalità anche con PyTorch Mobile ma non si dispone di tutte queste API che facilitano e velocizzano il processo.
    \item Manipolazione low-level del modello: al contrario di TensorFlow Lite, PyTorch è rinomato per la sua flessibilità nella manipolazione del modello\cite{pytorchvstensor}, offrendo la possibilità di personalizzare il modello fino ai minimi dettagli.
    \item Velocità di inferenza: una piacevole esperienza di utilizzo e creazione di un modello machine learning dipende molto anche dalla velocità con cui si riesce a testarlo e praticarlo. In questo contesto, TensorFlow Lite riesce a
    garantire un tempo di inferenza 3 volte più piccolo di quello di PyTorch Mobile grazie al supporto della GPU e dei suoi delegati (NNAPI).
    \item Dimensione: un’importante metrica da considerare nell’ambito della progettazione di modelli su dispositivi mobili è la dimensione del modello stesso. La capacità di memoria su questi dispositivi è minima e, per questo, c'è
    bisogno di un modello il più leggero possibile. Anche su questo fronte, TensorFlow Lite riesce a comprimere maggiormente la dimensione del modello garantendo un occupazione minima di memoria.
    \item Community e supporto: al fine di un uso più pacifico e sereno di un framework, avere una community fidata e del supporto ufficiale diventa essenziale. Sapere che c’è un team dedicato per segnalazioni di bug,
    richieste di nuove features e per un contributo attivo alla libreria è un fattore chiave. In questo contesto, TensorFlow Lite e PyTorch Mobile presentano entrambi una community attiva e una documentazione egualmente buona.
    Inoltre, sono entrambi open source.
\end{itemize}
In conclusione, TensorFlow Lite è la scelta migliore: maturità, supporto della GPU, dimensione delle librerie e del modello e larga gamma di API sono degli ottimi motivi per questa decisione.
PyTorch Mobile resta comunque un’opzione plausibile, soprattutto nei casi in cui è richiesta una personalizzazione profonda del modello e una flessibilità di controllo notevole.

    \fancypagestyle{plain}{
    \fancyhf{}
    \renewcommand{\headrulewidth}{0pt}
    }

    \begin{spacing}{1.8}                    %spaziatura tra un elemento e l'altro della bibliografia
      \printbibliography[heading=bibintoc]
    \end{spacing}

\end{document}
