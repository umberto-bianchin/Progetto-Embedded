\bchapter{Autenticazione e conversione del modello}
\section{Autenticazione}
L'autenticazione della nostra applicazione è stata gestita tramite Firebase Authentication\footnote{https://firebase.google.com/docs/auth?hl=it}, che
offre una soluzione robusta e sicura per la gestione degli utenti. Grazie a Firebase Authentication, gli utenti possono registrarsi, effettuare il login
e recuperare la password in modo semplice e intuitivo. Il processo di integrazione di Firebase Authentication nella nostra applicazione è stato facilitato
dalla disponibilità di SDK e API ben documentati, che hanno permesso di implementare le funzionalità di autenticazione in modo rapido ed efficiente.


\section{Conversione modello TensorFlow a TensorFlow Lite}
Sebbene per questa applicazione sia stato utilizzato un modello TensorFlow già in versione Lite, vogliamo comunque spiegare come TensorFlow offre una
soluzione leggera e veloce per l'esecuzione di modelli di machine learning su dispositivi con risorse limitate. Una volta che si dispone di un modello
addestrato (solitamente in formato ".h5"), è necessario installare la libreria di TensorFlow nel proprio ambiente Python. Dopo di che, con un semplice 
script (codice \ref{code:1}) è possibile convertire il modello in formato ".tflite":

\begin{code}
\begin{minted}[linenos]{python}
import tensorflow as tf
    
# Carico il modello TensorFlow (Keras)
model = tf.keras.models.load_model('modello.h5')

# Creo un'istanza di TFLiteConverter
converter = tf.lite.TFLiteConverter.from_keras_model(model)

# Converto il modello
tflite_model = converter.convert()

# Salvo il modello convertito
with open('modello.tflite', 'wb') as f:
    f.write(tflite_model)
\end{minted}
\caption{Script per conversione modello TensorFlow}
\label{code:1}
\end{code}
\bigskip

Una volta effettuata la conversione, è possibile ottimizzare il modello attraverso la quantizzazione (codice \ref{code:2}) o il pruning:

\begin{code}
\begin{minted}[linenos]{python}
# Abilitare la quantizzazione
converter.optimizations = [tf.lite.Optimize.DEFAULT]

# Convertire il modello con quantizzazione
tflite_quantized_model = converter.convert()

# Salvare il modello quantizzato
with open('modello_quantizzato.tflite', 'wb') as f:
    f.write(tflite_quantized_model)
\end{minted}
\caption{Script per quantizzazione del modello}
\label{code:2}
\end{code}


   
