\bchapter{TensorFlow}
TensorFlow è una libreria open source per l'apprendimento automatico, il calcolo numerico e altre attività di analisi statistica e predittiva.
Questo tipo di tecnologia, sviluppata e rilasciata da Google nel novembre 2015, rende l’implementazione di modelli di machine learning più semplice e
veloce per gli sviluppatori, assistendo nel processo di acquisizione dei dati, nella formulazione di previsioni su larga scala e nel successivo affinamento
dei risultati.

Lo scopo principale di TensorFlow è la creazione e l’addestramento di reti neurali, che possono essere utilizzate per moltissime applicazioni, quali:
\begin{itemize}
    \item Classificazione delle immagini;
    \item Elaborazione del linguaggio naturale;
\end{itemize}

\begin{figure}
    \centering
    \includegraphics[width=0.5\textwidth]{Immagini/logo_dei.png}
    \caption{Diagramma dei punteggi di utilizzo di vari framework nel 2018}
    \label{fig:diagramma}
\end{figure}
